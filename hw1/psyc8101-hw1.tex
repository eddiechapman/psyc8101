\documentclass[]{article}
\usepackage{lmodern}
\usepackage{amssymb,amsmath}
\usepackage{ifxetex,ifluatex}
\usepackage{fixltx2e} % provides \textsubscript
\ifnum 0\ifxetex 1\fi\ifluatex 1\fi=0 % if pdftex
  \usepackage[T1]{fontenc}
  \usepackage[utf8]{inputenc}
\else % if luatex or xelatex
  \ifxetex
    \usepackage{mathspec}
  \else
    \usepackage{fontspec}
  \fi
  \defaultfontfeatures{Ligatures=TeX,Scale=MatchLowercase}
\fi
% use upquote if available, for straight quotes in verbatim environments
\IfFileExists{upquote.sty}{\usepackage{upquote}}{}
% use microtype if available
\IfFileExists{microtype.sty}{%
\usepackage{microtype}
\UseMicrotypeSet[protrusion]{basicmath} % disable protrusion for tt fonts
}{}
\usepackage[margin=1in]{geometry}
\usepackage{hyperref}
\hypersetup{unicode=true,
            pdftitle={HW \#1: Review of Basic Statistical Concepts, Descriptive Statistics, \& Normal Distribution},
            pdfauthor={Eddie Chapman},
            pdfborder={0 0 0},
            breaklinks=true}
\urlstyle{same}  % don't use monospace font for urls
\usepackage{color}
\usepackage{fancyvrb}
\newcommand{\VerbBar}{|}
\newcommand{\VERB}{\Verb[commandchars=\\\{\}]}
\DefineVerbatimEnvironment{Highlighting}{Verbatim}{commandchars=\\\{\}}
% Add ',fontsize=\small' for more characters per line
\usepackage{framed}
\definecolor{shadecolor}{RGB}{248,248,248}
\newenvironment{Shaded}{\begin{snugshade}}{\end{snugshade}}
\newcommand{\KeywordTok}[1]{\textcolor[rgb]{0.13,0.29,0.53}{\textbf{#1}}}
\newcommand{\DataTypeTok}[1]{\textcolor[rgb]{0.13,0.29,0.53}{#1}}
\newcommand{\DecValTok}[1]{\textcolor[rgb]{0.00,0.00,0.81}{#1}}
\newcommand{\BaseNTok}[1]{\textcolor[rgb]{0.00,0.00,0.81}{#1}}
\newcommand{\FloatTok}[1]{\textcolor[rgb]{0.00,0.00,0.81}{#1}}
\newcommand{\ConstantTok}[1]{\textcolor[rgb]{0.00,0.00,0.00}{#1}}
\newcommand{\CharTok}[1]{\textcolor[rgb]{0.31,0.60,0.02}{#1}}
\newcommand{\SpecialCharTok}[1]{\textcolor[rgb]{0.00,0.00,0.00}{#1}}
\newcommand{\StringTok}[1]{\textcolor[rgb]{0.31,0.60,0.02}{#1}}
\newcommand{\VerbatimStringTok}[1]{\textcolor[rgb]{0.31,0.60,0.02}{#1}}
\newcommand{\SpecialStringTok}[1]{\textcolor[rgb]{0.31,0.60,0.02}{#1}}
\newcommand{\ImportTok}[1]{#1}
\newcommand{\CommentTok}[1]{\textcolor[rgb]{0.56,0.35,0.01}{\textit{#1}}}
\newcommand{\DocumentationTok}[1]{\textcolor[rgb]{0.56,0.35,0.01}{\textbf{\textit{#1}}}}
\newcommand{\AnnotationTok}[1]{\textcolor[rgb]{0.56,0.35,0.01}{\textbf{\textit{#1}}}}
\newcommand{\CommentVarTok}[1]{\textcolor[rgb]{0.56,0.35,0.01}{\textbf{\textit{#1}}}}
\newcommand{\OtherTok}[1]{\textcolor[rgb]{0.56,0.35,0.01}{#1}}
\newcommand{\FunctionTok}[1]{\textcolor[rgb]{0.00,0.00,0.00}{#1}}
\newcommand{\VariableTok}[1]{\textcolor[rgb]{0.00,0.00,0.00}{#1}}
\newcommand{\ControlFlowTok}[1]{\textcolor[rgb]{0.13,0.29,0.53}{\textbf{#1}}}
\newcommand{\OperatorTok}[1]{\textcolor[rgb]{0.81,0.36,0.00}{\textbf{#1}}}
\newcommand{\BuiltInTok}[1]{#1}
\newcommand{\ExtensionTok}[1]{#1}
\newcommand{\PreprocessorTok}[1]{\textcolor[rgb]{0.56,0.35,0.01}{\textit{#1}}}
\newcommand{\AttributeTok}[1]{\textcolor[rgb]{0.77,0.63,0.00}{#1}}
\newcommand{\RegionMarkerTok}[1]{#1}
\newcommand{\InformationTok}[1]{\textcolor[rgb]{0.56,0.35,0.01}{\textbf{\textit{#1}}}}
\newcommand{\WarningTok}[1]{\textcolor[rgb]{0.56,0.35,0.01}{\textbf{\textit{#1}}}}
\newcommand{\AlertTok}[1]{\textcolor[rgb]{0.94,0.16,0.16}{#1}}
\newcommand{\ErrorTok}[1]{\textcolor[rgb]{0.64,0.00,0.00}{\textbf{#1}}}
\newcommand{\NormalTok}[1]{#1}
\usepackage{graphicx,grffile}
\makeatletter
\def\maxwidth{\ifdim\Gin@nat@width>\linewidth\linewidth\else\Gin@nat@width\fi}
\def\maxheight{\ifdim\Gin@nat@height>\textheight\textheight\else\Gin@nat@height\fi}
\makeatother
% Scale images if necessary, so that they will not overflow the page
% margins by default, and it is still possible to overwrite the defaults
% using explicit options in \includegraphics[width, height, ...]{}
\setkeys{Gin}{width=\maxwidth,height=\maxheight,keepaspectratio}
\IfFileExists{parskip.sty}{%
\usepackage{parskip}
}{% else
\setlength{\parindent}{0pt}
\setlength{\parskip}{6pt plus 2pt minus 1pt}
}
\setlength{\emergencystretch}{3em}  % prevent overfull lines
\providecommand{\tightlist}{%
  \setlength{\itemsep}{0pt}\setlength{\parskip}{0pt}}
\setcounter{secnumdepth}{0}
% Redefines (sub)paragraphs to behave more like sections
\ifx\paragraph\undefined\else
\let\oldparagraph\paragraph
\renewcommand{\paragraph}[1]{\oldparagraph{#1}\mbox{}}
\fi
\ifx\subparagraph\undefined\else
\let\oldsubparagraph\subparagraph
\renewcommand{\subparagraph}[1]{\oldsubparagraph{#1}\mbox{}}
\fi

%%% Use protect on footnotes to avoid problems with footnotes in titles
\let\rmarkdownfootnote\footnote%
\def\footnote{\protect\rmarkdownfootnote}

%%% Change title format to be more compact
\usepackage{titling}

% Create subtitle command for use in maketitle
\providecommand{\subtitle}[1]{
  \posttitle{
    \begin{center}\large#1\end{center}
    }
}

\setlength{\droptitle}{-2em}

  \title{HW \#1: Review of Basic Statistical Concepts, Descriptive Statistics, \&
Normal Distribution}
    \pretitle{\vspace{\droptitle}\centering\huge}
  \posttitle{\par}
    \author{Eddie Chapman}
    \preauthor{\centering\large\emph}
  \postauthor{\par}
      \predate{\centering\large\emph}
  \postdate{\par}
    \date{September 10, 2019}


\begin{document}
\maketitle

\begin{enumerate}
\def\labelenumi{\arabic{enumi}.}
\tightlist
\item
  A psychologist records how many words participants recalled from a
  list under three different conditions: large reward for each word
  recalled, small reward for each word recalled, and no reward.

  \begin{enumerate}
  \def\labelenumii{\alph{enumii}.}
  \item
    What is the independent variable?

    \emph{Reward condition}
  \item
    What is the dependent variable?

    \emph{Word recall count}
  \item
    What kind of scale is being used to measure the dependent variable?

    \emph{Ratio scale}
  \end{enumerate}
\item
  Which of the following would be called a statistic and which would be
  called a parameter?

  \begin{enumerate}
  \def\labelenumii{\alph{enumii}.}
  \item
    The average income for 100 US citizens selected at random from
    various telephone books

    \emph{Statistic}
  \item
    The average income of citizens in the United States

    \emph{Parameter}
  \item
    The highest age among respondents to a sex survey in a popular
    magazine

    \emph{Statistic (generously)}
  \end{enumerate}
\end{enumerate}

\begin{quote}
Exercises \#3-6 are based on the following values for two variables, X
and Y: \[\begin{align*}
X_{1}&=2 & X_{2}&=4 & X_{3}&=6 & X_{4}&=8 & X_{5}&=10 \\
Y_{1}&=3 & Y_{2}&=5 & Y_{3}&=7 & Y_{4}&=9 & Y_{5}&=11
\end{align*}\]
\end{quote}

\begin{enumerate}
\def\labelenumi{\arabic{enumi}.}
\setcounter{enumi}{2}
\item
  By hand (you may use a calculator), find the value of each of the
  following expressions:
  \begin{align*}  a.\quad & \displaystyle\sum_{i=2}^{5} X_i & 4 + 6 + 8 + 10 &= 28\\  b.\quad & \displaystyle\sum 5X_i & (5 * 2) + (5 * 4) + (5 * 6) + (5 * 8) + (5 * 10) &= 150\\  c.\quad & \displaystyle\sum 3Y_i & (3 * 3) + (3 * 5) + (3 * 7) + (3 * 9) + (3 * 11) &= 105\\  d.\quad & \displaystyle\sum X_i^2 & 2^2 + 4^2 + 6^2 + 8^2 + 10^2 &= 220\\  e.\quad & (\displaystyle\sum Y_i)^2 & (3 + 5 + 7 + 9 + 11)^2 &= 1225\\  \end{align*}
\item
  Now, use R to evaluate all of the expressions from \#3.

\begin{Shaded}
\begin{Highlighting}[]
\NormalTok{X =}\StringTok{ }\KeywordTok{c}\NormalTok{(}\DecValTok{2}\NormalTok{, }\DecValTok{4}\NormalTok{, }\DecValTok{6}\NormalTok{, }\DecValTok{8}\NormalTok{, }\DecValTok{10}\NormalTok{)}
\NormalTok{Y =}\StringTok{ }\KeywordTok{c}\NormalTok{(}\DecValTok{3}\NormalTok{, }\DecValTok{5}\NormalTok{, }\DecValTok{7}\NormalTok{, }\DecValTok{9}\NormalTok{, }\DecValTok{11}\NormalTok{)}
\end{Highlighting}
\end{Shaded}

  \begin{enumerate}
  \def\labelenumii{\alph{enumii}.}
  \item
    \(\displaystyle\sum_{i=2}^{5} X_i\)

\begin{Shaded}
\begin{Highlighting}[]
\KeywordTok{sum}\NormalTok{(X[}\DecValTok{2}\OperatorTok{:}\DecValTok{5}\NormalTok{])}
\end{Highlighting}
\end{Shaded}

\begin{verbatim}
## [1] 28
\end{verbatim}
  \item
    \(\displaystyle\sum 5X_i\)

\begin{Shaded}
\begin{Highlighting}[]
\KeywordTok{sum}\NormalTok{(}\DecValTok{5} \OperatorTok{*}\StringTok{ }\NormalTok{X)}
\end{Highlighting}
\end{Shaded}

\begin{verbatim}
## [1] 150
\end{verbatim}
  \item
    \(\displaystyle\sum 3Y_i\)

\begin{Shaded}
\begin{Highlighting}[]
\KeywordTok{sum}\NormalTok{(}\DecValTok{3} \OperatorTok{*}\StringTok{ }\NormalTok{Y)}
\end{Highlighting}
\end{Shaded}

\begin{verbatim}
## [1] 105
\end{verbatim}
  \item
    \(\displaystyle\sum X_i^2\)

\begin{Shaded}
\begin{Highlighting}[]
\KeywordTok{sum}\NormalTok{(X}\OperatorTok{^}\DecValTok{2}\NormalTok{)}
\end{Highlighting}
\end{Shaded}

\begin{verbatim}
## [1] 220
\end{verbatim}
  \item
    \((\displaystyle\sum Y_i)^2\)

\begin{Shaded}
\begin{Highlighting}[]
\KeywordTok{sum}\NormalTok{(Y)}\OperatorTok{^}\DecValTok{2}
\end{Highlighting}
\end{Shaded}

\begin{verbatim}
## [1] 1225
\end{verbatim}
  \end{enumerate}
\item
  By hand (you may use a calculator), find the value of each of the
  following expressions: \begin{align*}
    a.\quad & \displaystyle\sum(X + Y) & (2 + 3) + (4 + 5) + (6 + 7) + (8 + 9) + (10 + 11) &= 65\\
    b.\quad & \displaystyle\sum XY & (2 * 3) + (4 * 5) + (6 * 7) + (8 * 9) + (10 * 11) &= 250\\
    c.\quad & (\displaystyle\sum X)(\displaystyle\sum Y) & (2 + 4 + 6 + 8 + 10) * (3 + 5 + 7 + 9 + 11) &= 1050\\
    d.\quad & \displaystyle\sum(Y - 2) & (3 - 2) + (5 - 2) + (7 - 2) + (9 - 2) + (11 - 2) &= 25\\
    \end{align*}
\item
  Now, use R to evaluate all of the expressions from \#5.

  \begin{enumerate}
  \def\labelenumii{\alph{enumii}.}
  \item
    \(\displaystyle\sum(X + Y)\)

\begin{Shaded}
\begin{Highlighting}[]
\KeywordTok{sum}\NormalTok{(X }\OperatorTok{+}\StringTok{ }\NormalTok{Y)}
\end{Highlighting}
\end{Shaded}

\begin{verbatim}
## [1] 65
\end{verbatim}
  \item
    \(\displaystyle\sum XY\)

\begin{Shaded}
\begin{Highlighting}[]
\KeywordTok{sum}\NormalTok{(X }\OperatorTok{*}\StringTok{ }\NormalTok{Y)}
\end{Highlighting}
\end{Shaded}

\begin{verbatim}
## [1] 250
\end{verbatim}
  \item
    \((\displaystyle\sum X)(\displaystyle\sum Y)\)

\begin{Shaded}
\begin{Highlighting}[]
\KeywordTok{sum}\NormalTok{(X) }\OperatorTok{*}\StringTok{ }\KeywordTok{sum}\NormalTok{(Y)}
\end{Highlighting}
\end{Shaded}

\begin{verbatim}
## [1] 1050
\end{verbatim}
  \item
    \(\displaystyle\sum(Y - 2)\)

\begin{Shaded}
\begin{Highlighting}[]
\KeywordTok{sum}\NormalTok{(Y }\OperatorTok{-}\StringTok{ }\DecValTok{2}\NormalTok{)}
\end{Highlighting}
\end{Shaded}

\begin{verbatim}
## [1] 25
\end{verbatim}
  \end{enumerate}
\item
  A veterinarian is interested in the life span of golden retrievers.
  She recorded the age at death (in years) of the retrievers treated in
  her clinic. The ages were \(12, 9, 11, 10, 8, 14, 12, 1, 9, 12\).

  \begin{enumerate}
  \def\labelenumii{\alph{enumii}.}
  \item
    Use R to calculate the mean, median, and mode for age at death.

\begin{Shaded}
\begin{Highlighting}[]
\CommentTok{# Return a vector's most frequently occurring value}
\CommentTok{#}
\CommentTok{# First, a new vector is created containing the input values }
\CommentTok{# with duplicates removed. The two vectors are compared to find}
\CommentTok{# the index positions where each input value occurs in the unique}
\CommentTok{# vector. The values (positions) in the index vector are counted}
\CommentTok{# for frequency and the position of the most frequent position }
\CommentTok{# value is returned. This is used to retrieve a value from the }
\CommentTok{# unique vector.}
\CommentTok{#}
\CommentTok{# https://stackoverflow.com/questions/2547402/is-there-a-built-}
\CommentTok{# in-function-for-finding-the-mode}
\NormalTok{Mode <-}\StringTok{ }\ControlFlowTok{function}\NormalTok{(x) \{}
\NormalTok{  ux <-}\StringTok{ }\KeywordTok{unique}\NormalTok{(x)}
\NormalTok{  ux[}\KeywordTok{which.max}\NormalTok{(}\KeywordTok{tabulate}\NormalTok{(}\KeywordTok{match}\NormalTok{(x, ux)))]}
\NormalTok{\}}
\end{Highlighting}
\end{Shaded}

\begin{Shaded}
\begin{Highlighting}[]
\NormalTok{dogs <-}\StringTok{ }\KeywordTok{c}\NormalTok{(}\DecValTok{12}\NormalTok{, }\DecValTok{9}\NormalTok{, }\DecValTok{11}\NormalTok{, }\DecValTok{10}\NormalTok{, }\DecValTok{8}\NormalTok{, }\DecValTok{14}\NormalTok{, }\DecValTok{12}\NormalTok{, }\DecValTok{1}\NormalTok{, }\DecValTok{9}\NormalTok{, }\DecValTok{12}\NormalTok{)}
\end{Highlighting}
\end{Shaded}

\begin{Shaded}
\begin{Highlighting}[]
\KeywordTok{mean}\NormalTok{(dogs)}
\end{Highlighting}
\end{Shaded}

\begin{verbatim}
## [1] 9.8
\end{verbatim}

\begin{Shaded}
\begin{Highlighting}[]
\KeywordTok{median}\NormalTok{(dogs)}
\end{Highlighting}
\end{Shaded}

\begin{verbatim}
## [1] 10.5
\end{verbatim}

\begin{Shaded}
\begin{Highlighting}[]
\KeywordTok{Mode}\NormalTok{(dogs)}
\end{Highlighting}
\end{Shaded}

\begin{verbatim}
## [1] 12
\end{verbatim}
  \item
    After examining her records, the veterinarian discovered that the
    dog that had died at 1 year was killed by a car. Using R,
    recalculate the mean, median, and mode without that dog's data.

\begin{Shaded}
\begin{Highlighting}[]
\NormalTok{dead_dogs <-}\StringTok{ }\KeywordTok{c}\NormalTok{(}\DecValTok{1}\NormalTok{)}
\NormalTok{dogs <-}\StringTok{ }\NormalTok{dogs[}\OperatorTok{-}\KeywordTok{which}\NormalTok{(dogs }\OperatorTok{==}\StringTok{ }\NormalTok{dead_dogs)]}
\NormalTok{dogs}
\end{Highlighting}
\end{Shaded}

\begin{verbatim}
## [1] 12  9 11 10  8 14 12  9 12
\end{verbatim}

\begin{Shaded}
\begin{Highlighting}[]
\KeywordTok{mean}\NormalTok{(dogs)}
\end{Highlighting}
\end{Shaded}

\begin{verbatim}
## [1] 10.77778
\end{verbatim}

\begin{Shaded}
\begin{Highlighting}[]
\KeywordTok{median}\NormalTok{(dogs)}
\end{Highlighting}
\end{Shaded}

\begin{verbatim}
## [1] 11
\end{verbatim}

\begin{Shaded}
\begin{Highlighting}[]
\KeywordTok{Mode}\NormalTok{(dogs)}
\end{Highlighting}
\end{Shaded}

\begin{verbatim}
## [1] 12
\end{verbatim}
  \item
    Which measure of central tendency in part \emph{b} changed the most,
    compared to the values in part \emph{a}? Why?

    \emph{The one-year-old dog was an outlier in the dataset. None of
    the other dogs lived less than 8 years. Removing this value affected
    the mean most of all, because it is sensitive to outliers. The
    median was also impacted because the removed value happened to be
    the minimum value. This change was not as dramatic.}
  \end{enumerate}
\item
  By hand, calculate the mean, sums of squares (\(SS\)), and variance
  (\(s^2\)) for the following sample of scores:
  \(11, 17, 14, 10, 13, 8, 7, 14\).

  mean: \(\frac{11 + 17 + 14 + 10 + 13 + 8 + 7 + 14}{8} = 11.75\)

  sums of squares:

  \((11 - 11.75)^2 + (17 - 11.75)^2 + (14 - 11.75)^2 + (10 - 11.75)^2 + (13 - 11.75)^2 + (8 - 11.75)^2 + (7 - 11.75)^2 + (14 - 11.75)^2 = 79.5\)

  variance: \(\frac{79.5}{7} = 11.357\)
\item
  Now using R, calculate all of the quantities from \#8.

\begin{Shaded}
\begin{Highlighting}[]
\NormalTok{scores <-}\StringTok{ }\KeywordTok{c}\NormalTok{(}\DecValTok{11}\NormalTok{, }\DecValTok{17}\NormalTok{, }\DecValTok{14}\NormalTok{, }\DecValTok{10}\NormalTok{, }\DecValTok{13}\NormalTok{, }\DecValTok{8}\NormalTok{, }\DecValTok{7}\NormalTok{, }\DecValTok{14}\NormalTok{)}
\end{Highlighting}
\end{Shaded}

\begin{Shaded}
\begin{Highlighting}[]
\KeywordTok{mean}\NormalTok{(scores)}
\end{Highlighting}
\end{Shaded}

\begin{verbatim}
## [1] 11.75
\end{verbatim}

\begin{Shaded}
\begin{Highlighting}[]
\KeywordTok{sum}\NormalTok{((scores }\OperatorTok{-}\StringTok{ }\KeywordTok{mean}\NormalTok{(scores))}\OperatorTok{^}\DecValTok{2}\NormalTok{)}
\end{Highlighting}
\end{Shaded}

\begin{verbatim}
## [1] 79.5
\end{verbatim}

\begin{Shaded}
\begin{Highlighting}[]
\KeywordTok{sum}\NormalTok{((scores }\OperatorTok{-}\StringTok{ }\KeywordTok{mean}\NormalTok{(scores))}\OperatorTok{^}\DecValTok{2}\NormalTok{) }\OperatorTok{/}\StringTok{ }\NormalTok{(}\KeywordTok{length}\NormalTok{(scores) }\OperatorTok{-}\StringTok{ }\DecValTok{1}\NormalTok{)}
\end{Highlighting}
\end{Shaded}

\begin{verbatim}
## [1] 11.35714
\end{verbatim}
\item
  If you convert each score in a set of scores to a \(z\)-score, which
  of the following will be true about the resulting set of \(z\)-scores?

  \begin{enumerate}
  \def\labelenumii{\alph{enumii}.}
  \tightlist
  \item
    The mean will equal 1.
  \item
    The variance will equal 1.
  \item
    The distribution will be normal in shape.
  \item
    All of the above.
  \item
    None of the above.
  \end{enumerate}
\item
  The SAT has a mean of 500 and a standard deviation of 100 in the
  population. What SAT score corresponds to

  \begin{enumerate}
  \def\labelenumii{\alph{enumii}.}
  \tightlist
  \item
    \(z= -0.2\)
  \item
    \(z= +1.3\)
  \item
    \(z= -3.1\)
  \item
    \(z= +1.9\)
  \end{enumerate}
\item
  Use the \(z\)-table to find the area under the normal distribution
  beyond \(z\) when \(z\) equals

  \begin{enumerate}
  \def\labelenumii{\alph{enumii}.}
  \tightlist
  \item
    \(+0.09\)
  \item
    \(+1.05\)
  \item
    \(+1.96\)
  \end{enumerate}
\item
  On a normal distribution, find the area between

  \begin{enumerate}
  \def\labelenumii{\alph{enumii}.}
  \tightlist
  \item
    \(z = -0.5\) and \(z = +1.0\)
  \item
    \(z = -1.5\) and \(z = +0.75\)
  \item
    \(z = +0.75\) and \(z = +1.5\)
  \end{enumerate}
\item
  Assume that the resting heart rate in humans is normally distributed
  with \(\mu = 72\)bpm (beats perminute) and \(\sigma = 8\)bpm.

  \begin{enumerate}
  \def\labelenumii{\alph{enumii}.}
  \tightlist
  \item
    What proportion of the population has resting heart rates above 82
    bpm?
  \item
    What proportion of the population has resting heart rates below 75
    bpm?
  \item
    What proportion of the population has resting heart rates between 80
    and 85 bpm?
  \end{enumerate}
\item
  A set of reading scores for fourth grade children has a mean of 25 and
  a standard deviation of 5. A set of scores for ninth grade children
  has a mean of 30 and a standard deviation of 10. Assume that the
  distributions are normal.

  \begin{enumerate}
  \def\labelenumii{\alph{enumii}.}
  \tightlist
  \item
    Draw a rough sketch of these data, putting both groups in the same
    figure.
  \item
    What percentage of the fourth graders score better than the average
    ninth grader?
  \item
    What percentage of ninth graders score worse than the average fourth
    grader?
  \end{enumerate}
\end{enumerate}


\end{document}
